% !TEX root = ../Projektdokumentation.tex
\section{Vorlesung 1}
\label{sec:Vorlesung1}
Montag: 08.April 2019

\subsection{Gruppen} 
\label{sec:gruppen}

Damit wir eine Gruppe bilden können, müssen folgende drei Regeln erfüllt werden.
\begin{eqnarray}
a * (b * c) = (a * b) * c  \\
id * a = a \\
a^{-1} * a = id
\end{eqnarray}
Dabei sei $\mathbb{M}$ eine Menge, in der IT endlich und *  eine Operation zwischen zwei Elementen der Menge $\mathbb{M}$. Man beachte, dass eine Gruppe nur das Kommutativ Gesetz erfüllt.

Des weiteren gilt in den betrachteten Gruppen:

\begin{eqnarray}
a * id  = id * a = a
\end{eqnarray}

da,
\begin{eqnarray}
a * id = a * (a^{-1} * a) \\
 =(a * a^{-1}) * a \\
\end{eqnarray}

und
\begin{eqnarray}
a * id &=& id * a = a\\
a &=& id * a | erweitert mit a^{-1} \\
a * a^{-1} &=& id * a^{-1} | mit (c * a^{-1} = id\\
&=&(c * a^{-1}) * (a * a^{-1}) \\
&=&((c * a^{-1}) * a) * a^{-1} \\
&=&( c * (a^{-1} * a)) * a^{-1} \\
&=&( c * (id)) * a^{-1} | id ist ein neutrales Element\\
&=&c * a^{-1} | mit c * a^{-1} = id \\
&=&id \\
\end{eqnarray}

Sei $\#$ die Menge der Elemente in $\mathbb{M}$. $\#$ wird auch als Leiterchen bezeichnet.

Sei $a^{m} = a^{n}$, dann ist die kleinst mögliche differenz zwischen den beiden $k$ und es gilt.

\begin{eqnarray}
a^{m} = a^{n} \\
a^{m+k} = a^{n} \\
a^{m} * a^{k} =  a^{n} \\
=> a^{k} = id
\end{eqnarray}

Sei $a * b = a * c$ ist $b = c$

Es gibt die Menge $\mathbb{M} = \{id, b_{1}, b_{2}, ...,  b_{\#-1}\}$ und die Menge $\mathbb{M}_0 = \{ id, a, a^{2}, a^{3}, ... , a^{k-1}\}$ in der es keine Dubletten gibt, das die erste Dublette $id = a^{k}$ wäre.

Wenn man nun neue Mengen bildet aus $\mathbb{M}$ und $\mathbb{M}_{0}$ gibt es $\#$ viele Mengen. Enthält eine Menge nur ein Element, dass auch in einer anderen Menge vorkommt, so sind beide Mengen identisch. 

Darauß ergibt sich, dass
\begin{eqnarray}
n * k = \# \\
a^{\#} = a^{kn} = (a^{k})^{n} = id^{n} = id
\end{eqnarray}


\subsection{Diffi-Hellmann}
\label{sec:hellmann}


\begin{table}[H]
\centering
\begin{tabular}{lll}
                                                                                                                                                      & Alice                      & Bob                        \\
\begin{tabular}[c]{@{}l@{}}\end{tabular} & M, *, \#                   & M, *,\#                    \\
                                                                                                                                                      & $g \in \mathbb{M}$                     & $g \in \mathbb{M}$ \\
private Key                                                                                                                                           & 1 \textless x \textless \# & 1 \textless y \textless \# \\
public Key                                                                                                                                            & $g^{x}$                         & $g^{y}$                          \\
Schlüsselvereinbarung                                                                                                                                 & $g^{y}$                         & $g^{x}$                          \\
                                                                                                                                                      & $(g^{y})^{x}$                        & $(g^{x})^{y}$                       
\end{tabular}%
\caption{Diffi-Hellmann - Schlüsselvereinbarung}
\label{diffiHellmanTable}
\end{table}

Dabei muss $\#$ muss genau bekannt sein. $g$ ist ein öffentliches Generatorpolynom. Und die Gruppe hat meist zwischen $2^{200}$ bis $2^{500}$ Werte. 

Wenn man einen Text verschlüsselt, wird jedes Zeichen des Textes verschlüsselt und übertragen in der Form $c_{i}g^{x_{i}y_{i}}$ und dann mit Hilfe des inversen entschlüsselt: $c_{i}g^{x_{i}y_{i}} * g^{-x_{i}y_{i}} = c_{i}$

Das inverse ist einfach $g^{\# - x}$.

Einschub: Es git eine Angriffsform die die Laufzeit untersucht, um anhand der Laufzeit und geschätzten Anzahl an Operationen Rückschlüsse auf den Verschlüsselungsalgorithmus zu machen.

\subsection{Gruppendarstellung und Untergruppen}
\label{sec:GruppenUtergruppen}

Man kann eine Menge $\mathbb{M} = \{ id, a, b\}$ als Tabelle aufschreiben.

\begin{table}[H]
\centering
\begin{tabular}{|l|l|l|l|}
\hline
*  & id & a  & b  \\ \hline
id & id & a  & b  \\ \hline
a  & a  & b  & id \\ \hline
b  & b  & id & a  \\ \hline
\end{tabular}
\caption{abstracte Menge mit 3 Elementen}
\label{Menge 3-Abstract}
\end{table}

Mann kann auch einfach die + Operation mod 3 in einer solchen Tabelle aufschreiben. Das wichtige ist, dass in einer Spalte und einer Zeile, jedes Element der Menge $\mathbb{M}$ nur einmal vorkommen darf.

\begin{table}[H]
\centering
\begin{tabular}{|l|l|l|l|}
\hline
+ mod 3 & 0 & 1 & 2 \\ \hline
0       & 0 & 1 & 2 \\ \hline
1       & 1 & 2 & 0 \\ \hline
2       & 2 & 0 & 1 \\ \hline
\end{tabular}
\caption{Menge 3 mit Zahlen}
\label{Menge 3-Addition}
\end{table}

Wenn man beide Tabellen vergleicht stellt man fest, dass sie isomorph sind. Das bedeutet, dass man einfach id = 0, a = 1 und b = 2 definieren kann und feststellt, dass die abstrakte Tabelle gleich der 3-Addition ist.

Des weiteren gilt, dass alle Gruppen die eine Größe $\#$ haben, die eine Primzahl sind haben genau eine Gruppe und besitzen keine Untergruppen. Daher sind alle Gruppen, die die selbe Größe $\#$ haben und diese Größe eine Primzahl ist isomorph.

Untergruppen entstehen wenn $\#$ keine Primzahl ist und so $\#$ einen Teiler hat. Jeder Teiler der durch Primfaktorzerlegung entsteht ist eine eigene Gruppe.

\begin{table}[H]
\centering
\begin{tabular}{|l|l|l|l|l|}
\hline
XOR & 0 & 1 & 2 & 3 \\ \hline
0   & 0 & 1 & 2 & 3 \\ \hline
1   & 1 & 0 & 3 & 2 \\ \hline
2   & 2 & 3 & 0 & 1 \\ \hline
3   & 3 & 2 & 1 & 0 \\ \hline
\end{tabular}
\caption{XOR Menge 4}
\label{xorMEnge4}
\end{table}

Diese Menge hat die Untergruppe:

\begin{table}[H]
\centering
\begin{tabular}{|l|l|l|l|l|}
\hline
XOR & 0 & 1 \\ \hline
0   & 0 & 1  \\ \hline
1   & 1 & 0 \\ \hline
\end{tabular}
\caption{Untergruppe Menge 2}
\label{Menge 2}
\end{table}
 
Untergruppen spielen eine wichtige Rolle bei der Wahl des Generatorpolynoms.

\subsection{Generatorpolynom}
\label{sec:generatorpolynom}

Das Generatorpolynom $g^{x}$ muss so gewählt werden, dass es eine möglichst große Teilmenge von $\mathbb{M}$ abdeckt. Wenn es nur einen kleinen Bereich abdeckt ist es leicht zu knacken.

Beispiel:
Sei die Primzahl 11 -> Rechnung mit mod 11

Vergleich von Generatorpolynomen:
\begin{table}[H]
\centering
\begin{tabular}{|l|l|l|l|l|l|l|l|l|l|l|}
\hline
     & 1  & 2 & 3 & 4 & 5  & 6 & 7 & 8 & 9 & 10 \\ \hline
g=2  & 2  & 4 & 8 & 5 & 10 & 9 & 7 & 3 & 6 & 1  \\ \hline
g=4  & 4  & 5 & 9 & 3 & 1  &   &   &   &   &    \\ \hline
g=10 & 10 & 1 &   &   &    &   &   &   &   &    \\ \hline
\end{tabular}
\caption{Generatorpolynom}
\label{generatorpolynom}
\end{table}

Man erkennt wenn man mit $g=2$ rechnet, dass man alle Elemente des Körpers durchgeht, bevor man auf das neutrale Element, die 1 stößt. Wenn man $g=4$ nimmt, dann werden nur noch 5 Elemente der Menge genutzt. Wenn man $g=10$ nutzt werden sogar nur 2 Elemente genutzt. Je weniger unterschiedliche Elemente genommen werden, desto einfacher ist es herauszubekommen welchen Wert x hat, sprich was der private Key ist.